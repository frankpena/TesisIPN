\chapter*{Introducción}
\addcontentsline{toc}{chapter}{Introducción}

Al someter a vibraciones netamente verticales a la superficie de un liquido, para un cierto rango de frecuencia y amplitudes en la superficie se presentan determinados patrones regulares, a este fenómeno se le llama \textit{Inestabilidad de Faraday}. El estudio de este tipo de inestabilidad tiene una larga e interesante historia que involucra a dos de los individuos más destacados de la ciencia moderna. A mediados del año 1831, M. Faraday (1791-1867) realizó una serie de experimentos con fluidos sometidos a vibraciones . El 19 de junio, informa en su diario la aparición de patrones cuadrados cuando una capa de fluido se pone en vibración. Trispaciones casi siempre cuadrangulares, siempre cuando están bien formadas, pero modificadas por el borde del agua o el líquido, también por el centro de movimiento '[19, §2]. Para examinar este fenómeno más de cerca, extendió el líquido en una placa horizontal, fijó la placa en el centro de una tira de vidrio o listón de soporte sostenido en los nodos y causó vibración por fricción. En particular, Faraday señaló que la frecuencia de las ondas era la mitad de la frecuencia del soporte. Medio siglo después, la respuesta subarmónica a las vibraciones puramente verticales fue cuestionada por L. Mathiessen [46], quien afirmó la sincronía del forzamiento y las ondas excitadas. Esta controversia motivó a Lord Rayleigh (1842-1919) a hacer su propio examen. En un primer experimento con una instalación como Faradays, una barra de hierro de un metro de largo se colocó en vibraciones mantenidas electromagnéticamente. Para medir las frecuencias, que eran demasiado altas (20 Hz) para ser reconocidas a simple vista, construyó un aparato que consiste en un disco de papel giratorio con uno, dos o cuatro agujeros. Observando la cantidad de imágenes de la barra y las olas, confirmó la declaración de Faradays. Para resolver la cuestión más allá de cualquier duda, proporcionó al bar pesas cerca del medio para ajustar la frecuencia de resonancia. Continúa: 'El tablón se puso en vibración mediante impulsos adecuadamente sincronizados con la mano, y los pesos se ajustaron hasta que el período correspondió a un modo de vibración libre del depósito de mercurio. Cuando se completa el ajuste, una vibración muy pequeña del tablón arroja el mercurio a una gran conmoción ... '[57]. Las frecuencias ahora se podían determinar directamente mediante inspección y, una vez más, el resultado apoyaba a Faraday.
La naturaleza subarmónica de la inestabilidad fue finalmente verificada teóricamente directamente de la hidrodinámica por Benjamin y Ursell [4], quienes en 1954 desarrollaron la teoría lineal. Ignorando la viscosidad, las ecuaciones fluidas ideales se expandieron en modos normales. Las amplitudes de los modos normales se desacoplan y cumplen como una primera aproximación de la ecuación de Mathieu, que es una ecuación diferencial ordinaria no autónoma de segundo orden. Benjamin y Ursell pudieron utilizar las propiedades de estabilidad conocidas de la ecuación de Mathieu para confirmar el punto de vista de Faraday y Rayleigh.
La inestabilidad encontrada en la ecuación de Mathieu se llama inestabilidad paramétrica, y es conocida por sistemas físicos tan diversos como osciladores electrónicos, ondas de Langmuir en plasma y yo-yos. El prototipo es el péndulo excitado paramétricamente. Si el pivote del péndulo oscila verticalmente, el péndulo comenzará a oscilar horizontalmente con la mitad de la frecuencia para algunas amplitudes y frecuencias. En este ejemplo, así como en el experimento de Faraday, la aceleración gravitacional efectiva es el parámetro externo impulsado. La ecuación de Mathieu lleva el nombre del matemático francés E. L. Mathieu (1825-1890). En su trabajo original de 1868 aparece al resolver la ecuación de onda bidimensional para el movimiento de una membrana elíptica.
Es interesante notar que Lord Rayleigh ya reconoció el experimento de Faraday como una inestabilidad paramétrica [56]. En su trabajo de 1883, en realidad analizó la ecuación de Mathieu en presencia de amortiguamiento y encontró una condición necesaria para la respuesta subarmónica. Más tarde [58] elaboró sobre el tema inspirado en el trabajo del físico estadounidense G. W. Hill (1838-1914) de 1877 sobre una generalización de la ecuación de Mathieu relativa al movimiento de la luna bajo la influencia del sol y la tierra.
En los años sesenta y setenta se realizó un trabajo para ampliar el análisis de Benjamin y Ursell a amplitudes finitas mediante la incorporación de no linealidades débiles. Sin embargo, con el creciente interés de la dinámica no lineal y el caos temporal en la última década, el experimento de Faraday ha ganado un interés más amplio. En 1981 Keolian et al. [37] observó un estado caótico en una célula anular fuertemente impulsada. Dos años después de que Gollub y Meyer [23] estudiaran la transición al caos de un modo único en una celda circular. Ciliberto y Gollub [7, 8] mostraron cómo la competencia entre dos modos superpuestos puede conducir al caos. Además, Simonelli y Gollub [62] estudiaron las interacciones de dos modos casi degenerados por simetría.
Los estudios experimentales fueron acompañados por esfuerzos teóricos [49, 31, 47, 26, 2, 21, 66]. El objetivo de estos estudios es extraer ecuaciones de amplitud para un solo modo o unos pocos modos resonantes de la hidrodinámica. Para una revisión ver Miles y Henderson [50].
La investigación mencionada anteriormente aborda el límite de las relaciones de aspecto bajas, es decir, cuando la longitud de onda es comparable al tamaño del sistema. En este caso, solo unos pocos modos se excitan simultáneamente y la dinámica es de baja dimensión. Gran parte del interés actual en el experimento de Faraday se debe a sus posibilidades como sistema con muchos grados de libertad. Para este propósito, el experimento de Faraday tiene dos ventajas principales; la relación de aspecto puede variarse simplemente variando la frecuencia, y la dinámica puede investigarse visualmente.
Un resultado sorprendente para las relaciones de aspecto altas es la observación de que el relieve de la superficie puede tomar la forma de patrones ordenados que se asemejan mucho a los cristales bidimensionales. Sin embargo, la verdadera sorpresa es que estos patrones no se limitan a las celdas de una geometría particular. Para relaciones de aspecto altas, los límites se vuelven menos importantes y la geometría es la del plano infinito. Esto ya lo notó Faraday. Al estudiar el patrón cuadrado observa: `Evidentemente, no es causado por ondas interferentes, aunque puede resolverse en ellas. ... Además, por irregulares que sean los bordes, la disposición puede hacerse cuadrangular '[19, §58]. Ezerskii et al. (1986) informaron recientemente un patrón cuadrado (1986) en una geometría circular. [14, 15]. En el mismo año, Aleksandrov et al. [1] encontró un patrón hexagonal para las amplitudes de la unidad debajo de aquellas para las cuales observaron el patrón cuadrado. La selección de patrones para las relaciones de aspecto intermedias ha sido investigada por Douady y Fauve [10, 11]. Notamos que se pueden introducir múltiples números de onda críticos a través de un forzado con más de un componente de frecuencia. Esto favorece el sistema Faraday de otros sistemas de formación de patrones como la convección de Rayleigh-Benard y permite la ingeniería de patrones [13].
Varios autores han estudiado experimentalmente el desglose del patrón cuadrado a un estado desordenado a medida que aumenta la amplitud de la unidad. Tuffilaro y col. [65] estudió el
longitud de correlación en función de la amplitud de la unidad y encontró un punto de transición bien definido. Por encima de la transición se encuentra el caos espacio-temporal o la turbulencia débil, es decir, se pierde la coherencia espacial pero la dinámica sigue dominada por una escala de longitud única. Ezerskii y col. [15, 16, 17] encontraron que la transición está mediada por el inicio de modulaciones transversales de longitud de onda larga. Las propiedades de transporte de las turbulentas olas de Faraday se han estudiado recientemente [59, 48]. Estos aspectos se revisan en las Refs. [22, 24] y [25].
Varios trabajadores han realizado intentos teóricos para comprender la formación del patrón y la transición al caos espacio-temporal. El más ambicioso parece ser el de Milner [51], que aplica el método de escalas múltiples para derivar las ecuaciones de amplitud. Sobre la base de las consideraciones energéticas, concluye que el estado preferido es el patrón cuadrado. Esto está en contradicción con la observación antes mencionada de un patrón hexagonal [1], y con los resultados experimentales reportados en la presente tesis. La teoría de Levin et al. [42] predice patrones hexagonales y cuadrados, pero sus métodos y presunciones parecen muy burdos. Ezerskii, Rabinovich y col. [15, 16, 54, 55] aplican principalmente su versión de las ecuaciones de amplitud al problema del caos e intermitencia espacio-temporal.
En esta tesis presentamos un estudio experimental del sistema de Faraday de alta relación de aspecto. Se hace hincapié en la observación de una serie de patrones cristalinos de diferente simetría. En particular, el descubrimiento de un estado cuasicristalino es de interés, ya que, según nuestro conocimiento, es el primer patrón de este tipo que se encuentra en un sistema hidrodinámico. El resto de la parte I de la tesis se organiza de la siguiente manera. En sec. 2 se desarrollan la hidrodinámica y la teoría lineal. La sección 3 ofrece una breve reseña histórica de los cuasicristales. En sec. 4 describimos la configuración experimental, y en la Sec. 5 se analiza el sistema óptico. En sec. 6 informamos nuestros resultados sobre la relación de dispersión, las tasas de amortiguamiento y la amplitud crítica, mientras que la Sec. 7 está dedicado a los patrones cristalinos y al diagrama de fases. La sección 8 contiene una discusión de la teoría no lineal. Cerramos esta parte de la tesis en la Sec. 9 con la conclusión.
