\chapter{Montaje Experimental}

3.1 Diseño experimental

      aquí habrá que hacer un dibujo del montaje y una descripción breve de los 
      materiales usados.

3.2 Patrones de Faraday en Reticulados

      describir los experimentos con celdas cuadradas, triangulares, y hexagonales.
      Los tamaños. el material. El problema del menisco, reticulado hidrofóbico, etc.
      Describir los parámetros importantes del experimento, los equipos y sus
      funciones. Los líquidos usados, lámparas, cámara, etc., el famoso shaker!!

3.3 Patrones de Faraday en Contenedores

      Decir que el montaje es el mismo. Aquí haremos experimentos con 2 capas de
      líquidos inmiscibles (aceite + agua). Tamaños de los contenedores, materiales,
      etc.

3.4 Gotas Suspendidas y Coalescencia de Gotas suspendidas

      ?  Esto lo vemos en el camino.

